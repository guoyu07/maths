\documentclass[a4paper,adobefonts]{book}

\usepackage{ctex}
\usepackage{mathrsfs}
\usepackage[headheight=12.5pt]{geometry}
\usepackage{makeidx}
\usepackage{fontspec}
\usepackage{xunicode}
\usepackage{xltxtra}
\usepackage{amsmath}
\usepackage{amssymb}
\usepackage{amsthm}

\graphicspath{{img/}}
\usepackage[svgnames,table]{xcolor}
\usepackage{booktabs}
\usepackage{longtable}
\usepackage{multirow}
\usepackage{paralist}


\usepackage[bookmarksnumbered,pdfencoding=auto,pdfauthor={穷屌丝联盟},pdfpagelayout=TwoPageRight,breaklinks,colorlinks,linkcolor=RoyalBlue,urlcolor=blue,colorlinks=true]{hyperref}
\usepackage{latexsym}
\usepackage{phonetic}
\usepackage{pstricks}
\usepackage{tikz}
\usepackage{setspace}
\usepackage{xlop}
\usepackage{polynom}
\usepackage{esint}
\usepackage{yhmath}
\usepackage{extarrows}
\usepackage{subfigure}
\usepackage[below]{placeins}

\usetikzlibrary{shapes.symbols}




\usepackage{fancyhdr}

\pagestyle{fancy}
\fancyhf{}
\fancyhead[LE,RO]{\thepage}
\fancyhead[RE]{\leftmark}
\fancyhead[RO]{\rightmark}
\fancypagestyle{plain}{
	\fancyhf{}
	\renewcommand{\headrulewidth}{0pt}
}

\setmainfont[Mapping=tex-text]{Adobe Garamond Pro}

\newtheorem{definition}{定义}[chapter]
\newtheorem{theorem}{定理}[chapter]
\newtheorem{lemma}[theorem]{引理}
\newtheorem{corollary}[theorem]{推论}
\newtheorem{property}[theorem]{性质}

\renewcommand{\thefootnote}{(\arabic{footnote})}

\renewcommand{\contentsname}{目录}
\renewcommand{\listfigurename}{图目录}
\renewcommand{\listtablename}{表目录}

\makeatletter
\newcommand\Wff{\mathop{\operator@font Wff}\nolimits}
\makeatother


\makeindex

\title{数学}
\author{穷屌丝联盟$\cdot$整理}
\date{\today}


\begin{document}
\maketitle
\tableofcontents
\listoffigures
\listoftables
\printindex

\zihao{5}

\renewcommand{\chaptermark}[1]{\markboth{#1}{}}

\include{whatismathematics}
%
%
%\makeatletter
%\@openrectotrue
%\makeatother
%\mainmatter
%\renewcommand{\chaptermark}[1]{\markboth{\chaptername: #1}{}}
%\renewcommand{\sectionmark}[1]{\markright{\thesection: #1}}
%
%
\include{number}
\include{number_theory}
\include{number_sets}

\include{geometry_algebra}

\include{sets_theory}

\include{logic}


%\include{Truth_Table}
%\include{Propositional_Formula}
%\include{Tautologies_Implications}
%\include{Other_Connectives}
%\include{Min_Connectives_Group}

\include{Mapping}
\include{Function}
\include{Function_Equal}


\include{Characteristic_Function}
\include{Countable_Set}

\include{Array}

\include{Math_Function}

\include{Derivative}

\include{Function_Graphics}

\include{Monotonicity}
\include{Boundedness}
\include{Parity}
\include{Periodicity}
\include{Function_Calculation}
\include{Elementary_Function}
\include{Limit}
\include{Limit_Computation}

\include{Special_Limits}

\include{Limits_Decision}
\include{Two_Important_Limits}
\include{Function_Continuity}


\include{Nature_of_Continuous_Function}
\include{Operation_Nature_of_Continuous_Function}
\include{Function_Uniform_Continuity}
\include{Discontinuity_Point}


\include{Derivative_and_Differential}
\include{Derivative_Rule}
\include{High_Derivatives}

\include{Differential_Mean_Value_Theorem}
\include{Differential}
\include{Differential_Rule}
\include{Application_of_Differential}


\include{Function_Extremum}

\include{Arc_Differential}

\include{Curvature}
\include{Approximate_Solution_of_Equation}
\include{Integration}
\include{Indefinite_Integral}

\include{Definite_Integral}
\include{Double_Integral}
\include{Basic_Formula_of_Calculus}
\include{Calculation_of_Definite_Integral}
\include{Space_Graphics_and_Curve}
\include{Double_Integral_Calculation}
\include{Double_Integral_Methods}
\include{General_Integral}

\include{Numeric_Integral}
\include{Variable_Integral}
\include{Curve_Surface_Integral}






\end{document}




